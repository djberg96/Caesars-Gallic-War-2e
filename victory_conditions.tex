\section{Victory Conditions}
\par
The Barbarian player wins instantly if Caesar is killed. Otherwise, victory is checked at the end of turn 8 (51 BC).

If the Roman player has scored at least 90 victory points, then the Roman player has scored a Minor Victory. Caesar returns to Rome, receives a Triumph and avoids prosecution owing to his popularity.

If the Roman player has scored at least 100 victory points, then the Roman player has scored a Major Victory. History repeats itself. Caesar takes a legion into Italy in violation of Imperium and a Roman Civil War begins.

If the Roman player has scored less than 90 victory points then Barbarian player is considered victorious. Caesar returns to Rome without an army and is prosecuted for crimes during his consulship.

In addition to the cumulative end of turn VP scored, the following events also affect the Roman player's VP score:

\begin{itemize}
  \setlength\itemsep{0em}
  \item Each German non-leader unit eliminated: +1 VP
  \item Ariovistus eliminated: +2 VP
  \item Vercingetorix eliminated: +3 VP
  \item Amphibiously invade Britannia with at least 2 legions and fight at least one round of combat: +2 VP*
  \item Invade Germania with at least 2 legions and fight at least one round of combat and/or force at least one German unit to withdraw: +2 VP*
  \item Each Gallic home unit in Britannia eliminated: +1 VP (max 2 VP)*
  \item Each Roman legion unit eliminated: -5 VP
\end{itemize}

Note that the points for invading Britannia and Germania can only be scored once during the game. The number of VP for killing Gallic home units in Britannia is restricted to a maximum of 2 VP.

* Ignore this if playing with the optional Yearly Objectives.