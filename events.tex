\clearpage
\section{Events}
\subsection{Baggage Train}
If played by the Roman player, add 5 supply points to the Roman
Supply Track. If played by the Barbarian player, either distribute
2 steps among friendly barbarian units in any one area, or subtract 2
supply points from the Roman Supply Track.

\subsection{Revolts - General Rules}
The following rules apply to all types of revolts (Minor, Major and Massive).

Germania may never be the target of a Revolt event.

The act of a tribe returning to its home area is not considered movement, i.e. it does not affect border crossing limits, retreat restrictions, etc. It is simply picked up and placed in its home area. If it is already in its home area it remains there.

If the area of a tribe returning home is occupied by units of the opposing player then a battle is resolved immediately. The returning unit becomes the attacker and the opposing units already there become the defender. For Major and Massive Revolt events, the Barbarian player determines the order of resolution if there is more than one battle.

A neutral tribe activated by a revolt does not count against the Neutral Tribe Activation limit.

For revolt purposes, when selecting a tribe, you are selecting all tribes of that area, e.g. if you select the Bellovaci, you are also selecting the Caletes.

\subsubsection{Exceptions}

If the area of one or more units returning home has a fort, they may retreat into the fort as their action during combat, if possible. Units of the opposing player must then either attack as per the siege rules, or regroup to one or more friendly-controlled adjacent areas, if possible.

The Helvetii tribe may not be the target of a Revolt on Turn 1 (58 BC).

German units may never be the target of a Revolt.

You may not select a tribe if all units of its area have already been eliminated this turn.

If the home area of the tribe selected is home to two different tribes and one has been eliminated, then only the one still on the board is affected. Any units already eliminated in the current turn do \textbf{not} return to the board because of a Revolt.

\subsection{Minor Revolt}
If played by the Roman player, he selects a single active tribe controlled
by the Barbarian player. All units of that tribe's area return at their current strength to their home area immediately, and are now controlled by the Barbarian player. If their home area is not occupied by any other units, they revert to neutral status at full strength.

If the tribe’s home area is occupied by Roman or Barbarian units, the tribe joins the other player (i.e. the player who does not have units in that tribe’s home area), and a battle is resolved immediately.

If played by the Barbarian player, he selects a single active tribe
controlled by the Roman player or a neutral tribe. All units of that
tribe's area (active or neutral) return at their current strength to their home area immediately and become Barbarian allies.

\subsection{Major Revolt}
If played by the Roman player treat as a Minor Revolt.

If played by the Barbarian player, select two active tribes controlled by the Roman player, neutral tribes, or a combination of the two. At least one of these areas must not be in Aquitania. If controlled by the Roman player, all units of that tribe immediately return to their home area at their current strength, and are now controlled by the Barbarian player.

If battle occurs as the result of tribes returning home and all Gallic tribe units are eliminated then the event has no further effect, i.e. there is no minor leader placement, even if all Roman units were eliminated as well.

If the home areas of the selected tribes do not contain any Roman units, or battle results in a Barbarian victory, then place a minor leader in one of the selected areas. The leader should match the color of the area selected, i.e. Ambiorix if the home area is in Belgica, or Dumnorix if the home area is in Celtae. The area in which the minor leader is placed is now considered the home area for that leader until the leader is eliminated.

The Barbarian player may then immediately activate those areas (only) for movement and possible combat.

If both minor leaders are already on the map, the Barbarian player treats this event as a Minor Revolt.

\subsection{Massive Revolt}
If played by the Roman player treat as a Minor Revolt.

If played by the Barbarian player on turn 1 treat as a minor revolt.

If played by the Barbarian player on turn 2 or later, then the Barbarian player selects any three areas, whether neutral or controlled by the Roman player. If controlled by the Roman player, all tribes of those areas immediately return to their respective home areas at their current strength. All the units of the selected areas are now controlled by the Barbarian player.

If battle occurs and all Gallic tribe units are eliminated then the event has no further effect, i.e. there is no leader placement, even if all Roman units were eliminated as well. Otherwise, place the Vercingetorix unit in any of the areas of the four selected tribes. If battle occurred, he may only be placed where there was a Barbarian player victory.

The area in which the Vercingetorix unit is placed is now considered his home area. However, if his home area is later controlled by the Romans then the Barbarian player may immediately select any friendly controlled
Gallic area as his new home area.

After placing Vercingetorix, the Barbarian player may immediately activate the any three areas for movement and possible combat. He is not limited to the three tribal areas that were selected for the event. 

If the Vercingetorix unit is eliminated it is removed from the game permanently.

Once this card has been played as a Massive Revolt by the Barbarian player it is removed from the game. Do not remove it from the game if played by either player as a Minor Revolt or a movement action.