\section{Yearly Objectives}
This is an optional rule designed to steer players towards the actual historical results. The sections below are broken down per year, with a title briefly describing the major events and/or historical figures of that year.

At the end of each turn at the end of the Score Victory Points phase, after scoring points normally, the Roman player may gain or lose additional victory points as the result of certain conditions. Those conditions are dependent on the turn, and are described below. It is possible that more than one condition applies, in which case the Roman player should gain bonus VP first, then apply lost VP afterwards.

\subsection{Turn 1 (58 BC): Ariovistus}
\begin{itemize}
  \setlength\itemsep{0em}
  \item The Roman player controls Sequani, Allobroges and Helvetii/Nantuates territory at the end of the turn: +1 VP
  \item The Barbarian player controls Sequani territory at the end of the turn: -1 VP
  \item The Helvetii unit has not been destroyed by the end of the turn: -1 VP
\end{itemize}

\subsection{Turn 2 (57 BC): The Belgic Campaign}
\begin{itemize}
  \setlength\itemsep{0em}
  \item The Roman player controls at least 3 territories adjacent to Oceanus Britannicus: +1 VP
  \item The Roman player controls the Remi/Aduatuci territory: +1 VP
  \item The Barbarian player controls at least 3 territories adjacent to Oceanus Britannicus: -1 VP
\end{itemize}

\subsection{Turn 3 (56 BC): Galba, Crassus and the Venetic Campaign}
\begin{itemize}
  \setlength\itemsep{0em}
  \item The Roman player controls the Helvetii/Nantuates territory and garrisons at least one legion there at the end of the turn: +1 VP
  \item The Roman player controls both the Osismi and Veneti territories: +1 VP
  \item The Roman player controls at least one tribe in Aquitania: +1 VP
  \item The Barbarian player controls the Helvetii/Nantuates territory at the end of the turn: -1 VP
  \item The Barbarian player controls at least one tribe in Aquitania: -1 VP
  \item The Barbarian player controls the Helvetii/Nantuates territory: -1 VP
\end{itemize}

\subsection{Turn 4 (55 BC): The German Campaign and First Expedition to Britannia}
\begin{itemize}
  \setlength\itemsep{0em}
  \item The Barbarian player controls the Menapi/Nervii territory and winters at least one German unit there at the end of the turn: -1 VP
  \item The Barbarian player controls the Atrebates/Morini territory and winters at least one German unit there at the end of the turn: -1 VP
  \item The Roman player crosses into Germania with at least one Roman legion and fights at least one round of combat: +1 VP
  \item The Roman player crosses into Britannia with at least one Roman legion and fights at least one round of combat: +1 VP
\end{itemize}

\subsection{Turn 5 (54 BC): Dumnorix, The British Campaign, and Winter Quarters}
\begin{itemize}
  \setlength\itemsep{0em}
  \item The Roman player crosses into Britannia with at least two Roman legions and fights at least one round of combat: +1 VP
  \item The Roman player kills Dumnorix: +1 VP
  \item The Roman player winters with at least one legion in at least four different territories at the end of the turn: +1 VP
  \item The Barbarian player controls Britannia at the end of the turn: -1 VP
  \item The Barbarian player has Dumnorix in play at the end of the turn: -1 VP*
  
*The Barbarian player must bring Dumnorix into play as the first card played this turn in order for this to apply.
\end{itemize}

\subsection{Turn 6 (53 BC): The Treveri Campaign, Ambiorix, and a raid into Germania}
\begin{itemize}
  \setlength\itemsep{0em}
  \item The Roman player controls the Treveri/Eburones, Remi/Aduatuci, and Menapi/Nervii territories at the end of the turn: +1 VP
  \item The Roman player kills Ambiorix: +1 VP
  \item The Roman player crosses into Germania with at least one legion: +1 VP
  \item The Barbarian player controls the Treveri/Eburones, Remi/Aduatuci and Menapi/Nervii territories at the end of the turn: -1 VP
  \item The Barbarian player has Ambiorix in play at the end of the turn: -1 VP*
  
*The Barbarian player must bring Ambiorix into play as the first card played this turn in order for this to apply.
\end{itemize}

\subsection{Turn 7 (52 BC): Vercingetorix}
\begin{itemize}
  \setlength\itemsep{0em}
  \item The Roman player kills Vercingetorix: +2 VP
  \item The Barbarian player controls Vercingetorix and at least four Gallic tribes at the end of the turn: -2 VP*
  \item The Barbarian player has not brought Vercingetorix into play by the end of the turn: +1 VP
  
The Barbarian player must bring Vercingetorix into play as the first card played this turn in order for this to apply.
\end{itemize}

\subsection{Turn 8 (51 BC): Mopping Up}
\begin{itemize}
  \setlength\itemsep{0em}
  \item The Roman player has subjugated at least one tribe in each of Belgica, Aquitania and Celtae: +2 VP
  \item The Barbarian player controls more tribes than the Roman player: -2 VP
\end{itemize}