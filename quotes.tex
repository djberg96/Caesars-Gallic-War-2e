\section{Quotes}
\par

\blockquote[Book I, Chapter 23]{I was no more than 18 miles from Bibracte, by far the largest and richest oppidum of the Aedui. I decided the next day that we should do something about the grain supply. So I stopped following the Helvetii and marched for Bibracte.}

\blockquote[Book I, Chapter 31]{However, things had turned out worse for the victorious Sequani than for the vanquished Aedui: Ariovistus, the king of the Germans, had settled in their territory and had seized a third of their land, the best in the whole of Gaul.}

\blockquote[Book I, Chapter 33]{Then too I saw that it was dangerous for Rome to have the Germans gradually getting into the habit of crossing the Rhine and coming into Gaul in vast numbers.}

\blockquote[Book III, Chapter 11]{I ordered Publius Crassus to set out for Aquitania with twelve cohorts of legionaries and a large number of cavalry, to stop reinforcements being sent from those tribes to Gaul, and to prevent such powerful peoples from joining forces.}

\blockquote[Book III, Chapter 13]{After capturing several of their oppida, I realized that all of this effort was being wasted. Even when we had done so, it was impossible to stop the enemy from getting away by ship, or to do them any real damage. So, I decided I must wait for our fleet to arrive.}

\blockquote[Book IV, Chapter 5]{When I was informed about these events I felt uneasy because of the temperament of the Gauls. They are always ready to change one plan for another and in general are always eager for political change, and I thought I ought not to rely on them.}

\blockquote[Book IV, Chapter 16]{With the German war concluded, I decided that I must cross the Rhine. Several reasons prompted me. The strongest was that I could see the Germans were all too ready to cross into Gaul, and I wanted them to have reasons of their own for anxiety when they realized that an army of the Roman people could and would cross the Rhine.}

\blockquote[Book V, Chapter 24]{The grain harvest that year in Gaul had been poor because of drought, so I was compelled to change my usual methods of arranging winter quarters for my legions, and distribute them among a larger number of tribes.}

\blockquote[Book V, Chapter 26]{On returning to Gaul from Britain in the autumn of 54 BC Caesar was faced with a crisis. Lack of grain forced him to site the winter quarters of his legions more widely throughout Belgic territories than he would otherwise have chosen.}

\blockquote[Book V, Chapter 53]{Caesar sent Fabius back to his winter quarters with his own legion, while he himself decided to spend the winter with three legions, each in its own camp, around Samarobriva. Since such formidable Gallic uprisings had occurred, he thought it best to remain personally near the army for the entire winter.}

\blockquote[Book VII, Chapter 14]{He therefore called his supporters to a council of war, and pointed out to them that the war must be waged in quite a different way from before. They must direct all their efforts towards cutting the Romans off from forage and supplies.}