\section{Leaders}
\par
A few units represent significant historical figures at this time. They are integrated into units, rather than having separate units. Some leaders have special abilities.

\subsection{Julius Caesar}
\par
Legion X (the tenth legion) represents Julius Caesar. Caesar’s unit always activates first regardless of the initiative rating of defending units or other special rules. In addition, Caesar’s unit does not count against an area’s garrison limit.

If Legion X is eliminated (i.e. Julius Caesar is killed) then the Barbarian player instantly wins an automatic victory.

\subsection{Ariovistus}
\par
Ariovistus was the king of the Germans (Suebi). His name is written on his unit to distinguish him from other German units. He begins the game in Germania, which is his home area.

When Ariovistus attacks a neutral area, first roll a die for each neutral Gallic unit. On a 1-2, that tribe joins the Barbarian player automatically and no battle is fought against that unit. The Barbarian player may then regroup normally. If the Ariovistus unit is eliminated (i.e. Ariovistus is killed), then he is removed from the game permanently.

Note that it is possible that, in an area with multiple tribes, one Gallic unit could join the Barbarian player while the other does not. In that case, the tribe that joined the Barbarian player would be considered part of the main attacking force against the remaining unit.

\subsection{Vercingetorix}
\par
Vercingetorix appears as a result of the play of the Massive Revolt event card by the Barbarian player and he is always allied to the Barbarian player.

The Barbarian player receives a -1 die roll modifier on all Political Action attempts if the attempt is made against a tribe whose home area is adjacent to the Vercingetorix unit. If this ability is used then the Barbarian player must announce the location (but not strength) of Vercingetorix. This ability is optional. The Barbarian player may forfeit the bonus and resolve a Political Action normally in order to avoid revealing the location of Vercingetorix.

Vercingetorix may remain outside of his home area every turn. He is not obligated to return to his home area. Any Gallic tribe (but not German) units in the same area as Vercingetorix at the end of the turn may remain in the same area as Vercingetorix, so long as they don’t exceed the area’s garrison limit.

\textbf{Note:} This is the only time Gallic tribe units may remain outside of their home area at the end of the turn.

While Vercingetorix is in play the Barbarian player may conduct one Supply Action (raid) per turn. If Vercingetorix is eliminated, he is removed from the game permanently.

\subsection{Minor Gallic Leaders}
\par
There are two minor Gallic leader units, Dumnorix and Ambiorix, which can appear as the result of the play of the Major Revolt event card by the Barbarian player. They are always allied to the Barbarian player. Minor leaders have no special abilities. They are merely an extra unit for the Barbarian player.

Minor Gallic leaders may never be controlled by the Roman player. Although the minor leaders have been given names for historical flavor, they are not permanently eliminated from the game if removed during play. A minor leader could be eliminated and later return through the play of another Major Revolt card. There are two home area markers for the minor leaders to place and keep track of where their home areas are.

\textit{These leaders abstractly represent various Gallic leaders that would crop up from time to time and stir up trouble against the Romans.}
