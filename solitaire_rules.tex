\section{Solitaire Rules}

If you wish to play this game solitaire, then you will play as the Romans and attempt to match Caesar's historical accomplishments. Some changes to setup apply, as well as special rules that will govern how the Barbarian "bot" behaves.

Since you are playing solitaire, you may play with your units face up if you wish. However, to avoid potentially confusing your own units with enemy controlled units, all bot units should remain hidden and pointed away from you until combat occurs.

Setup: Only deal cards to yourself, the Roman player. The Barbarian bot will be drawing cards, one at a time, from the top of the deck over the course of the turn. The action it takes will depend on certain factors, described below.

Card Action Phase: You, as the Romans, play your card normally as per the standard two-player rules. If you initiate combat, or if the Barbarian units attack you, then you will be rolling for both sides. Some special rules governing battles in the solitaire game are provided below.

After you play your card and complete your action, draw a card from the top of the deck for the Barbarian bot and reveal it, setting it to the side. The following action matrix determines what the bot does, in this order:

1) If the card revealed is a neutral area, and the bot has not yet activate 2 neutral areas yet, then treat it as a neutral tribe activation. The tribes in the area on the card are now controlled by the bot.

2) If the card revealed is a neutral area, but the bot has already activated 2 neutral areas, then treat it as a political action. Roll to resolve the political action, using the -1 bonus for the card matching the area. If successful, the tribes in that area are now controlled by the bot.

3) If the card revealed shows the name a subjugated tribe, and that tribe's home area is not occupied by enemy units, then the tribe is unsubjugated. Return it, face down, to its home area. If the card reveals an area that is home to two tribes, and both of its tribes have been subjugated, select one randomly.

3a) If the subjugated tribe's home area is occupied by enemy units, then roll a die instead. On a 1-3, treat this card as a political action as per option 2 above, and randomly select a tribe as the target. On a 4-6 activate the Germania area, select Ariovistus and one other unit (highest strength possible) and attack the nearest neutral or enemy controlled area. If there are multiple neutral or enemy controlled areas, choose the one with the fewest blocks. If multiple areas have the same number of fewest blocks, then choose one randomly.

4) If the card revealed shows an area that is already controlled by the bot, then that area is activated for movement. All tribes in that area will attack an adjacent target area that contains only one block, whether Roman or Barbarian. If there is more than one, then select one randomly. Non-neutral areas should be prioritized over neutral areas.

4a) If there are no adjacent areas with only one block, and the activated area contains at least two blocks, then instead they will attack an adjacent target area that contains only two blocks. Again, determine one randomly if there is more than one.

If a river prevents all units from attacking the target area, then only two randomly selected units will attack, and the rest will remain.

* If there are no adjacent units with only one or two blocks

RANDOM TRIBE SELECTION

FIRST ROLL (2d6):

Roll 2d6 to determine the region that is selected - Aquitania, Belgica or Celtae. The Celtae region is split into three parts - East, West, and Central.

2-3: Aquitaine
4-5: Celtae (West)
6-7: Celtae (East)
8-9: Celtae (Central)
10-12: Belgica

SECOND ROLL:

Roll 1d6 to determine the specific tribal area.

Aquitania:

1-3: Tarbelli
4-6: Tolosates

Belgica:

1: Atrebates
2: Bellovaci
3: Mediomatrici
4: Menapi
5: Remi
6: Treveri

Celtae East:

1: Aedui
2: Roll again, 1-3: Allobroges, 4-6: Helvetii
3: Boii
4: Leuci
5: Mandubii
6: Sequani

Celtae West:

1: Osismi
2: Veneti
3: Santones
4: Pictones
5: Andes
6: Cadurci

Celtae Central:

1: Arverni
2: Bituriges
3: Britannia
4: Carnutes
5: Esuvii
6: Volcae

