\section{Card Phase}
\subsection{Draw Cards}
Shuffle the card deck and deal five cards to each player, shuffling in any cards played from previous turns first.

\textbf{Exception:} Only deal four cards to each player on the first turn of the game.

If the Romans have scored at least ??? number of victory points, and at least 13 Gallic tribes are Roman allied, the Barbarian player has the option of putting the Massive Revolt card in his hand. If this option is used, only deal four additional cards to the Barbarian player.

If not already played, at the start of 52 BC (only), the Barbarian player may select the Massive Revolt card, even if the Romans do not meet the minimum VP and/or allied Gallic tribes. However, if this option is used, the Roman player receives 1 VP instantly. In addition, it must be the first card played by the Barbarian player, and must be played as an event.

\subsection{Card Action Phase}
Each player selects one card from their hand and places that card face down. Both players then simultaneously reveal which card they have selected. Each player then announces what action they are taking, Romans first. Actions are then resolved in card action order.

There are possible actions and they are resolved in the following order: Event, Supply Action, Neutral Tribe Activation, Political Action, or Movement. If both players select the same action, the Romans resolve their action first.

\textbf{Exception:} If both players selected a Movement action, the player who played a card with the highest action value moves first. If both players played a card with the same action value, the Romans move first.

At the end of each card play, i.e. both players have played their selected card and completed their action, a battle occurs in each area where both players have units they control in the same area. See the "Battle" rules section for more details.

This process is repeated until all cards have been played and resolved. Played cards should be placed into the discard pile unless used for a Neutral Tribe Activation, in which case they should be placed in the player's respective Neutral Tribe Activation area on the map.

Cards cannot be held between turns, i.e. a player may not pass in order to save a card for a future turn. Discard piles are not secret, and may be examined at any time by either player.

All played cards and the unused portion of the deck are shuffled together at the end of the turn for use at the start of the next turn.

\subsubsection{Card Action Phase Order}

\textbf{A. Events}
\par
There are five event cards in the game. Each event card may be played for the event as described on the card, or may instead be played as a 1 for a Movement Action. Event cards may never be used for a Political Action, Neutral Tribe Activation, or Supply Action. The cards explain the event and how it is applied to the game for each player. See the card roster section at the back of the rules for a more detailed explanation of each event.

\textbf{B. Supply Action}
\par
The Roman player tracks supply points on the General Records track on the game board using the Roman Supply block. The Roman player may add a number of supply points equal to double the Action Point value of the card.

\textit{Example: The Roman player uses the Germania card (3 action points) for a supply action. The Roman player adds 6 supply points (3 x 2) to the supply track.}

The Barbarian player may not use a card for a Supply Action until the leader Vercingetorix is in play. The Barbarian player's version of a supply action is to \textbf{reduce} the number of Roman supply points by the Action Point value of the card. Unlike the Roman player, this value is NOT doubled.

The Barbarian player may perform a maximum of one Supply Action per year, whereas the Roman player can perform as many Supply Actions as desired.

\textbf{C. Neutral Tribe Activation}
\par
Both players may activate neutral tribes and automatically bring them under their control as allies at full strength. The name of the home area on the card must match the name of neutral tribe's home area that the active player wishes to activate. When neutral tribes are activated, stand the unit(s) upright, facing the player to which they are now allied. In areas that have two tribes, both tribes are activated.

Neither player can change control of a tribe that is already controlled by the opposing player with a Neutral Tribe Activation. The Barbarian player is limited to two Neutral Tribe Activations per year, while the Roman player is limited to one Neutral Tribe Activation per year.

\textbf{Exception:} The Roman player may not use a Neutral Tribe Activation to activate any units in Britannia or Germania.

\label{political_action}\textbf{D. Political Action}
\par
Either player may use a card to attempt to change the political allegiance of any Gallic tribe on the board, whether it's neutral or controlled by the opposing player.

Roll a die. If the selected area of the Political Action matches the home area on the card, the active player receives a -1 bonus to the die roll. If the inactive player controls a unit that is in that area, then the active player receives a +1 penalty to the die roll.

If the die roll is less than or equal to the Action Point value on the card played, the Political Action is successful and control of that area is switched to the active player. The Gallic tribe's block is immediately returned to its home area (if not already home) at its current strength. If there are enemy units present then a battle is immediately fought (see 5.2.2 Battles). The units returning home are considered the attacker.

If the area is home to multiple units, then a successful political action is considered successful against all units of that area. They all return home as a single force.

If a Political Action is successful against the home area of a Minor Leader, that leader is immediately removed. If a Political Action is successful  against the home area of Vercingetorix, the Barbarian player immediately designates a new home area for him. The newly selected area must belong to a tribe that is currently controlled by the Barbarian player and is free of enemy units. If no such area exists, then Vercingetorix is eliminated. The Roman player scores VP normally in this case.

\textbf{Exceptions:} Neither the Roman player nor the Barbarian player may target Britannia as the target of a Political Action unless there is at least one Roman legion (Romans), or a leader (Barbarians), in a port area connected to Oceanus Britannicus. The Roman player may not use a Political Action against Germania.

A neutral tribe activated by a Political Action does not count against the Neutral Tribe Activation limit.

\label{movement}\textbf{E. Movement}
\par
The active player may activate for movement a number of groups up to the Action Point value of a card, or none at all if they choose. All units within an area area considered a group. Each group moving may move together to the same area or individual units within the group could move to separate areas. Unit movement rate is one area per card play. Retreat and regroup do not count as movement. All groups must be designated prior to actual movement.

\textit{For example, 4 Roman legions starting in Transalpine Gaul could as part of one group move have 2 units move into Allobroges and 2 units move into Helvetii, or all 4 into Allobroges, or any number of other moves within the restrictions of movement.}

\textbf{Exception:} Each Roman legion unit in the off-map area is considered its own group, i.e. they must be moved onto the map one at a time, and only into Transalpine Gaul. Barbarian units may not enter the off-map area.
\par
Borders and other terrain restrict movement and limit the number of units that may move from one area to another adjacent one. A maximum of four units may cross a black border from one area to another. A maximum of two units may cross a blue border from one area to another. Border limits are applied separately to each player. For example, both the Roman and Barbarian player could move two units across the same blue border in the same card play.

All units crossing a blue border, moving using naval movement, or which enter an area containing enemy or neutral units must stop immediately. This supersedes forced marches by Roman units, i.e. a supply point may not be spent to force march a Roman unit that has just crossed a blue border (though it could cross a blue border on its second move).

Units moving into an area containing enemy or neutral units are considered to be the attacker. Units already in an area which the opposing player moves units into are considered the defender. Entering an area that contains a neutral tribe on the board causes that tribe to join the opposing player, though battle is not resolved until the end of the card play round. If both players move into the area of a neutral tribe, the tribe will join the player who moved into the area last.

\textit{\textbf{Pinning:}} When moving into an area, all attacking units (including reserves, see below) prevent an equal number of defending enemy units from moving in that round. The defender chooses which units are pinned. Any unpinned units may move and attack normally, except that they may not cross any border used by the attacker to enter that area.

\textit{\textbf{Roman Unit Force March:}} Roman legion units have the option to expend one supply per unit in order to conduct a forced march. This allows the Roman legion to move up to two areas per card play instead of one.

\textbf{CROSSING THE RHINE}
\par
Only German units and Roman legion units may cross the Rhine. Gallic tribes may not cross the Rhine.

\textbf{Exception:} Roman units may not cross the Rhine into Germania if there are no German units there.

\textbf{NAVAL MOVEMENT}
\par
An ocean area may not be entered or crossed except from port area to port area using naval movement. Only units that start in a port area may use naval movement. Up to two units may move from one port area to another port area within the same sea zone as part of one group move. Units that start in the Osismi area may move across either sea zone since it is considered to have a port in both sea zones.

Barbarian units may not use naval movement to enter a port that contains enemy or neutral units. Conversely, Roman units may use naval movement to enter a port area that contains neutral or enemy units at the cost of 1 supply point per group. Note that this affects the battle by treating the units defending the port area as having "A" initiative, and limiting the battle to two battle rounds instead of the normal three.

\textbf{STACKING AND ATTRITION}
\par
There is no limit to the number of blocks that may occupy an area on the map. However, if a player has more than 6 blocks in an area at the end of their card play, then each block in excess of 6 takes one point of attrition damage.

\textbf{Exception:} Roman legions do not count against the stacking limit in Transalpine Gaul. There is no stacking limit in the Roman off-map area.

The owning player may choose which units take step losses, but may not eliminate a block if possible, and Gallic non-leader units must take losses prior to Roman, German or Gallic leader units.

\textbf{Movement Example}
\par
\textit{The Roman player used a '1' action point card for movement, and may thus activate one group (area). He activates the group in Transalpine Gaul since his six legions are all there. He moves legion VII and VIII directly from Transalpine Gaul into the Helvetii area. Since it's a blue border he cannot move any more units across that border. He then uses two supply points to force march legions IX and X (Caesar) from Transalpine Gaul to Allobroges, and from there into Helvetii. Roman supply is reduced from 15 to 13.}
\par
\textit{The Romans then force march legions XI and XII from Transalpine Gaul to Sequani through Allobroges. Roman supply is reduced from 13 to 11. The Roman legions can force march through Allobroges because the Gallic units in Allobroges are Roman allies. If the Roman player had played a '2' action point card, he could also have moved the Roman allies in Allobroges or any other group in an area.}
