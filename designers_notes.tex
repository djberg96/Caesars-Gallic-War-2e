Differences between first and second editions of Caesar's Gallic War.

* A new map.
  - Additional areas.
  - Two sea zones instead of one.
  
* New units
  - There is now greater variability between the units, both in terms of strength and initiative rating.
  - Areas that used to contain multiple blocks with the same tribe now have two distinct tribes, potentially with different values.
  - Some areas will start with different units at the start of each game, creating more variability between games.
  - The Helvetii are replaced by the Nantuates if eliminated. 
  
* Victory Points
  - Now uses a cumulative VP system instead of a fixed VP level.
  - Yearly objectives added for flavor.
  
* Movement
  - The maximum number of units that can cross a black border has been reduced from six to four.
  
* Combat
  - Now includes a fortress/siege system to replace the simple bonuses that existed before.
  - Attacks across the Rhine, as well as amphibious invasions, are now restricted to two rounds of combat instead of the normal three.
  - German units can now strategically withdraw (i.e. leave the game) instead of getting farmed for points.
  
* Supply
  - Cards are now doubled for Romans (instead of value + 1).
  - Romans use supply to recruit reinforcements instead of fixed arrivals.
  - The cost to supply Roman legions in the field is now fixed at 1.
  - The harvest roll now only affects garrison limits, it does not affect the cost to garrison.
  
* End of turn
  - German and Gallic replacements happen before eliminated units are returned to the map. This fixes a quirk of the old system where eliminated units would be back nearly to full strength right away.
  
* Solitaire
  - A much more formal set of solitaire rules have been added.