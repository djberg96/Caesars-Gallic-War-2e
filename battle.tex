\section{Battle}
\par
Battles and/or sieges are fought as a result of Event, Political Action and Movement card play. Each battle must be completed before fighting the next battle. Battles occurring as the result of Event and Political Action card play are resolved immediately with the order determined by the active player. The player who selected a movement action and moved first this round determines the order in which battles are fought as the result of a movement card play.

\textbf{A. Deployment}
\par
In any area in which there is a battle both players reveal their units by tipping their blocks forward so that they show their current strength. If a fortress is present, after all units are revealed, the defender must decide if units are to be deployed in the field to fight a battle or in the fortress, up to the limit of the fortress. In most cases this means a maximum of one unit, except Transalpine Gaul, which can hold up to two units. It is thus possible that a defending unit is inside the fortress while other defending units are in the field.

If a unit is selected to go into a fortress, the first unit selected must be a tribe in its home area. In the case of Transalpine Gaul, Roman legion units must be selected first. Blocks that are inside the fortress may not retreat, but may opt join the field battle.

\textbf{B. Battle Rounds}
\par
Each battle/siege is fought over a maximum of three combat rounds. This can be three rounds of battle, three rounds of siege, or any combination of those, such as two battle rounds and 1 siege round.

In battles, the attacker must retreat all blocks at the end of the third round of combat if there any defending units still in the field. In sieges, the attacker may retreat or stay in siege.

Each unit may fire, retreat or pass during each battle round. This is called a battle action. The order in which a unit resolves its battle action depends on its initiative. Any defending units with an "A" initiative take their battle action first, followed by all attacking "A" units. Then all defending "B" units take their action, followed by all attacking "B" units. Finally, all defending "C" units take their action, followed by all attacking "C" units.

\textbf{Exception:} If the Caesar unit (legion X) is involved in a battle, it always takes its action first, regardless of enemy initiative or circumstances.
