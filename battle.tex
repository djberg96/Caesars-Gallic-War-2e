\section{Battle}
\par
Battles are fought as a result of Event, Political Action and Movement card play. Each battle must be completed before fighting the next battle. Battles occurring as the result of Event and Political Action card play are resolved immediately with the order determined by the active player. The player who selected a movement action and moved first this round determines the order in which battles are fought as the result of a movement card play.

In any area in which there is a battle both players reveal their units by tipping their blocks forward so that they show their current strength. After each battle is completed, return any surviving units upright before proceeding to the next battle. After all units have taken one battle action, one battle round is considered to have passed. Repeat the sequence for a 2nd and 3rd round if necessary.

Most battles are fought for a maximum of 3 battle rounds, though when crossing the Rhine into Germania or amphibiously invading an area there are only 2 battle rounds. The attacker must retreat if the defender is not eliminated and does not retreat at the end of the maximum number of battle rounds. This procedure is repeated until all battles are complete.

Each unit may fire, retreat or pass during each battle round. This is called a battle action. The order in which a unit resolves its battle action depends on its initiative. Any defending units with an "A" initiative take their battle action first, followed by all attacking "A" units. Then all defending "B" units take their action, followed by all attacking "B" units. Finally, all defending "C" units take their action, followed by all attacking "C" units.

\textbf{Exception:} If the Caesar unit (legion X) is involved in a battle, it always takes its action first, regardless of enemy initiative or circumstances.

Each unit may “fire” by rolling as many dice as its strength. A hit is scored for every roll less than or equal to its battle rating. Against most units each hit reduces an enemy unit by one strength point. This is indicated by rotating the unit counter-clockwise for each hit. Enemy units may not be targeted individually. Each hit is applied to the strongest enemy unit. If two or more enemy units are the same strength, the owner decides on which unit to apply the hit. Units eliminated in battle are set aside until the End of Turn Phase.

\textbf{Exceptions:} In mixed forces, Roman or German units must take a hit before allied Barbarian units if they have the same strength.

Some units require two hits to be reduced by one strength. If possible, all hits must be applied. If an odd number of hits remain at the end of a battle round after applying all hits, the excess are ignored, except in sieges. See below.

\textbf{A. Reserves}

Multiple groups may attack or defend the same area, moving across the same or different borders. However, only one group is considered the main group. All other groups are considered reserves. In the case of defending units, the units that began the action in an area are considered the main group. Any units that moved to help defend the area are considered reserves. In the case of attacking units, the players attacking an area decides which group is the main group and which groups are reserves.

Reserve units may not perform a battle action or suffer hits, nor are they revealed, in the first round of battle. They are revealed and participate in the 2nd round of battle, even if all other friendly units have been eliminated. If the attacking player eliminates all defending units before defending reserve units arrive, then the original attacker is now considered the defender, and the original defender is now considered the attacker. Battle is then resolved normally.

\textbf{B. Sieges}
\par
Defending units that start a battle in an area with a fortress may declare that they are inside the fortress, up to the limit of the supply value of the fortress. Thus, most fortresses may only hold one unit, while Transalpine Gaul can hold up to two units. Defending reserve units do not have this option.

If a Gallic unit is defending in its home area with a fortress, then that unit must be chosen first over other potential units to defend inside a fortress, if any are chosen at all.

Units defending in a fortress are treated as having an "A" initiative. In addition the defending unit has double defense, i.e. it requires two hits to score one point of damage. Unlike most units, a unit under siege retains leftover hits until the end of battle.

Defending units inside a fortress may not retreat or reorganize after battle unless the attacking units retreated or were eliminated first.

\textit{Designer's Note: Historically the Roman sieges of the various Gallic fortified towns were very brief, and were usually resolved by Caesar cutting off his enemy's food and water supply. Once that was accomplished, the barbarians would soon surrender.}

\textit{The largest and longest siege of the campaign at Alesia lasted just weeks, while most seem to have been completed in mere days. For this reason I do not provide siege rules that are more detailed because they simply wouldn't be appropriate at this scale.}

\textbf{C. Roman Units Crossing the Rhine}
\par
When the Romans cross the Rhine into Germania the ensuing battle is restricted to a maximum of two rounds instead of the normal three. Every German unit killed in battle immediately scores 1 VP for the Roman player, except Ariovistus, which scores 2 VP for the Roman player.

\textbf{Germanic Strategic Withdrawal Option}
\par
The Barbarian player may strategically withdraw any or all German units in Germania, including Ariovistus, from the board before battle begins. This is not like a retreat, the units are not moved to a friendly area. Instead they are removed from the game.

Every German unit that strategically withdraws does not score any VP for the Roman player, but it is then permanently removed from the game.

Note that the strategic withdrawal option is only available for German units in Germania. It does not apply to German units fighting outside of Germania.

\textit{Designer's Note: This is designed to simulate the strategic value of Caesar's raids across the Rhine. Historically, these did not cause much in the way of actual damage or casualties because the Germanic tribes withdrew rather than risk a pitched battle. What it did accomplish, however, was to demonstrate to the Germans that the Romans could cross the Rhine at any time, and effectively convince them not to cross the Rhine into Gaul again.}
\par
\textit{Thus, the Barbarian player has to make a choice whether to risk giving the Roman player more VP's through a pitched battle, or to minimize the Roman VP potential at the cost of permanently losing control of German units.}

\textbf{D. The Alps}
\par
Units defending in the Alps receive double defense. In other words, it requires two hits in battle to score one point of damage. A unit that already has double defense does not receive any additional benefit.

\textbf{E. Retreat}
\par
Instead of firing, each unit may retreat to an adjacent friendly or empty area. Units may retreat to the same or different areas. Retreating units are returned to their face up position before retreating, potentially concealing which units are retreating to any particular area. Border limits apply to retreating units on a per-battle-round basis, e.g. four units could retreat across a blue border, but no more than 2 per battle round

Units may not retreat into an area through which enemy units entered the area. However, if both players moved units across the same border in the same round, only the player who crossed the border last may retreat across that border.

Units may not retreat into a contested area, i.e. an area where there is an unfought battle. Only German units may retreat into Germania.

\textbf{Naval Retreat}
\par
Roman units conducting a naval move into an enemy port may retreat to any friendly port area in the same sea zone. They may not retreat if there is no friendly port area in the same sea zone. Any units defending in a port area may retreat up to two units (maximum, not per battle round) to another friendly port area.

\textbf{E. Regroup}
\par
The player left controlling the area at the end of battle is considered the victor and may regroup surviving units. Regroup allows all victorious units, including any in reserve, to move to any adjacent friendly or empty area.

A Unit cannot regroup into an unfought battle. Only German units may regroup into Germania. Border limits apply to regrouping units.
